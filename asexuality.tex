
% Preamble
    \documentclass{article}
    \usepackage{arxiv}
    
    \usepackage[utf8]{inputenc} % allow utf-8 input
    \usepackage[T1]{fontenc}    % use 8-bit T1 fonts
    \usepackage{hyperref}       % hyperlinks
    \usepackage{url}            % simple URL typesetting
    \usepackage{booktabs}       % professional-quality tables
    \usepackage{amsfonts}       % blackboard math symbols
    \usepackage{nicefrac}       % compact symbols for 1/2, etc.
    \usepackage{microtype}      % microtypography
    \usepackage{lipsum}
    
    \linespread{1}

% /Preamble

\title{
    Sexuality in the Church \\ 
    \large Historical and Contemporary Investigation}
% /title

\author{
  Asher Mancinelli \\
  Department of Computer Science\\
  Eastern Washington University\\
  Spokane, WA 99218 \\
  \texttt{asher.mancinelli@pnnl.gov} \\ }
% /author

\begin{document}
\maketitle

\begin{abstract}
    Here is my abstract for the paper. Here is some more filler text.
\end{abstract}

% Consider using the keywords section to define terms that I do not 
% undersrtand well. This will help me communicate well with the reader
% even if we have different understandings of terms. We will have to come
% to terms, and since this is a one-way form of communication I have to be 
% the one to define terms, lest ambiguity and misunderstanding permeate the
% paper
% \keywords{\bold{Asexuality}: def \and Second keyword \and More}

\linespread{1}

\section{Introduction}
    % Consider writing this after the outline has been finished
    This is the introduction section
    Broad investigation into the history of sexuality in the church, with
    special attention paid to asexuality.

\section{First Century: In the Wake of Judaism}

In both Judaism and Christianity, context is the primary determinant in 
the interpreted morality of any sexual action, and the methods of interpreting
context vary greatly in either faith. Here we compare and contrast the 
methods of determining morality in either faith in the first century. The
reader should mind the fact that either fatih has had a plethora of opinions on
sexuality over the last two millenia, and this section only examines some sects
of Judaism and the early church in the first century. This narrow investigation
does not necessarily reflect contemporary opinion.

\subsection{Jewish Theology on Sexuality in the $1^{st}$ Century}
    Christianity was born out of Jewish culture, therefore Jewish theology
    of sexuality is a necessary primer for understanding Christian theology
    of sexuality. Of the four major sects of Judaism in the first century
    (Pharisees, Sadducees, Essenes and Zealots), Jesus's theology most
    resembled the Pharisees\cite{finkelstein_1929}, so it is their 
    opinions that I have given the most attention to.
    The Jewish theology of sexuality within the scope of this research
    may be encapsulated by the following:
    \begin{itemize}
        \item{The First Mitzvah}
        \item{Levitical Law of Holiness}
        \item{Pharisaical Perspective on Sex Apart from the Law}
    \end{itemize}

    \subsubsection{The First Mitzvah}
        Genesis 1:28?

    \subsubsection{Levitical Law of Holiness}
    \subsubsection{Pharisaical Perspective on Sex Apart from the Law}

\subsection{Pauline Theology on Sex}

    Paul represents an interesting theology: he was formerly the most
    devoted of the Jews, deeply passoinate and educated on the Law (in
    the royal Jewish sense, not the social or legal sense) before being
    converted to Christianity. 
    % TODO: scriptural reference needed here
    And, as Jesus declares in the Gospel of Matthew, Chapter 5, no part
    of the Law passes away in Christianity. Thus, Paul's expertise in the
    Law is applied to the New Covanent under Christ that the New
    Testament preaches.

\subsection{Monasticism and Sex}
\subsection{Reformation: Sex is Back in Fasion}
\subsection{Postmodernism and the Liberal Christian}
\subsection{The $21^{st}$ Century Christian}

\bibliographystyle{unsrt}  
\bibliography{references}

\end{document}
