
% Preamble
    \documentclass{article}
    \usepackage{arxiv}
    
    \usepackage[utf8]{inputenc} % allow utf-8 input
    \usepackage[T1]{fontenc}    % use 8-bit T1 fonts
    \usepackage{hyperref}       % hyperlinks
    \usepackage{url}            % simple URL typesetting
    \usepackage{booktabs}       % professional-quality tables
    \usepackage{amsfonts}       % blackboard math symbols
    \usepackage{nicefrac}       % compact symbols for 1/2, etc.
    \usepackage{microtype}      % microtypography
    \usepackage{lipsum}
    \usepackage{setspace}

    \setlength{\parindent}{4em}
    \setlength{\parskip}{1em}
    
    \linespread{1}

% /Preamble

\title{
    Sexuality in the Church \\ 
    \large Historical and Contemporary Investigation}
% /title

\author{
  Asher Mancinelli \\
  Department of Computer Science\\
  Eastern Washington University\\
  Spokane, WA 99218 \\
  \texttt{asher.mancinelli@pnnl.gov} \\ }
% /author

\begin{document}
\maketitle

\begin{abstract}
\end{abstract}

% Consider using the keywords section to define terms that I do not 
% undersrtand well. This will help me communicate well with the reader
% even if we have different understandings of terms. We will have to come
% to terms, and since this is a one-way form of communication I have to be 
% the one to define terms, lest ambiguity and misunderstanding permeate the
% paper
\keywords{\textbf{Asexuality}: those who experience no sexual attraction
    or desire \and \textbf{the Church}: the royal sense of the church;
    the global church across time since it's inception \and \textbf{the Early
    Church Fathers}: the most active leaders in the church during the 
    \textit{Patristic era}}
%/keywords

\doublespace

\section{Introduction}
    % Consider writing this after the outline has been finished
    This is the introduction section Broad investigation into the 
    history of sexuality in the church, with special attention paid 
    to asexuality.  Note that any Scriptural reference will be taken 
    from the \textit{English Standard Version} of the Bible unless 
    otherwise specified. Note also that I do not make any claim to a complete
    understanding of the historical church - it is no secret that there is
    underrepresentation of women in the history of the church, and I do
    not anticipate that my research will be able to compensate for this.
    I will however do my best to find resources on women in the church
    through history.

\section{First Century: In the Wake of Judaism}

    In both Judaism and Christianity, context is the primary determinant in 
    the interpreted morality of any sexual action, and the methods of 
    interpreting context vary greatly in either faith. Here we 
    compare and contrast the 
    methods of determining morality in either faith in the first century. The
    reader should mind the fact that either fatih has had a plethora 
    of opinions on sexuality over the last two millenia, and this 
    section only examines some sects of Judaism and the early church in 
    the first century. This narrow investigation does not necessarily 
    reflect contemporary opinion.

\subsection{Jewish Theology on Sexuality in the 1$^{st}$ Century}
    Christianity was born out of Jewish culture, therefore Jewish theology
    of sexuality is a necessary primer for understanding Christian theology
    of sexuality. Of the four major sects of Judaism in the first century
    (Pharisees, Sadducees, Essenes and Zealots), Jesus's theology most
    resembled the Pharisees\cite{finkelstein_1929}, so it is their 
    opinions that I have given the most attention to.
    The Jewish theology of sexuality within the scope of this research
    may be encapsulated by the following:
    \begin{itemize}
        \item{The First Mitzvah}
        \item{Levitical Law of Holiness}
        \item{Pharisaical Perspective on Sex Apart from the Law}
    \end{itemize}

    \subsubsection{The First Mitzvah}
        Genesis 1:28?

    \subsubsection{Levitical Law of Holiness}
    \subsubsection{Pharisaical Perspective on Sex Apart from the Law}

\subsection{Pauline Theology on Sex}

    \par Paul represents an interesting theology: he was formerly the most
        devoted of the Jews, deeply passoinate and educated on the Law (in
        the royal Jewish sense, not the social or legal sense) before being
        converted to Christianity. 
        % TODO: scriptural reference needed here
        And, as Jesus declares in the Gospel of Matthew, Chapter 5, no part
        of the Law passes away in Christianity. Thus, Paul's expertise in the
        Law is applied to the New Covanent under Christ, which the New
        Testament preaches. \\
    %/par

    \par Readers from a Western culture must pay special attention to
        translation in interpreting Pauline writings however, becuase for
        hundreds of years, traslations into Western languages have been shaping
        our beliefs. This is the case with other languages as well, but there
        is a unique attachement to certain translations in Western culture,
        especially in Baptist sects in the United States. \\
    %/par

    \par This is seen especially in 1 Cor. 7:1. The King James Version (KJV)
        reads: \textit{Now concerning the things whereof ye wrote unto me: 
        It is good for a man not to touch a woman.} This may perhaps lead
        the reader to believe that Paul would advocate for total abstinence
        and an abandonement of any sexual activity. However, this would be
        a clear misinterpretation of Paul's intent, and a meager attemt at
        exegesis. In other translations, like the English Standard Version,
        the subject of the sentence is quoted for clarity: \textit{Now
        concerning the matters about which you wrote: "It is good for a
        man not to have sexual relations with a woman".} It is clear here
        that Paul is directly addressing what the church in Corinth had
        written him previously\cite{kjv_sex}, and that the former's
        interpretation is unhelpful and potentially hurtful. \\
    %/par

    \par If we continue to read Paul's letter, he \textit{advises and compells}
        those in the church to marry, and even specifically admonishes them
        to have sex regularly as a part of the love and mutual submission
        of a healthy relationship:
        \par \textit{Do not deprive one another, except perhaps by agreement
        for a limited time, that you may devote yourselves to prayer; but
        then come together again, so that Satan may not tempt you because
        of your lack of self control.} 
        \par \textit{Now as a concession, not a command, I say this.
        I wish that all were as I myself am. But each has his own
        gift from God, one of one kind, and one of another.}
        (1 Cor. 7:5, \textit{ESV}) \\
        \par I am struck by the fact that so many hold a stigma in Western
        Christian culture against asexual folk, those not sexually
        interested in anyone, when the most substantive writer of the 
        New Testament wished that all had his \textit{gift} of seemingly
        being asexual: He did not command that all should refrain from sex,
        rather that we should be especially careful to have regular sex with
        our marital partners. How I wish that the church would more
        carefully examine the Scripture before passing a value judgement
        on one who does not experience sexualy attraction; they
        consequently miss the fact that Paul willed that they themselves
        would be gifted with asexuality, that they may more completely
        devote themselves to the Kingdom - and asexuality is a gift. \\
    %/par

    \par To conclude the section on Paul's theology, we draw two conclusions:
        \begin{itemize}
            \item{translation affects interpretation, thus the church ought
                to strive for the most accurate translation possible, and}
            \item{Paul was of the opinion that sex is a gift, and
                asexuality in its own right.}
        \end{itemize}
        I am sure that had Paul not suffered the heuristical injustice of not
        having language or concept of asexuality, he would have ensured that
        asexuality be recognized as a gift, of greater or equal value to
        that of any form of sex.
    %/par

\subsection{Early Church Fathers}
    
    \par The void of resources on women in this time period must be recognized
        up front. The fact that the time period is recognized as the \textit{
        Patristic era} should serve as some indicator that female representation
        in this research will be lacking.
    %/par

    \par With that recognized, we now look to the Early Church Fathers'
        theology of sex. It would seem that, as in every era, the Church
        Fathers made some concessions to their contemporaries in their
        theology. 
    %/par


\section{Monasticism and Sex}
\section{Reformation: Sex is Back in Fasion}
\section{Postmodernism and the Liberal Christian}
\section{The 21$^{st}$ Century Christian}

\bibliographystyle{unsrt}  
\bibliography{references}

\end{document}
