
% Preamble
    \documentclass[12pt]{article}
    \usepackage{arxiv}

    \usepackage[utf8]{inputenc} % allow utf-8 input
    \usepackage[T1]{fontenc}    % use 8-bit T1 fonts
    \usepackage{hyperref}       % hyperlinks
    \usepackage{url}            % simple URL typesetting
    \usepackage{nicefrac}       % compact symbols for 1/2, etc.
    \usepackage{microtype}      % microtypography
    \usepackage{setspace}

    \setlength{\parindent}{4em}
    \setlength{\parskip}{1em}
    
    \linespread{1}

% /Preamble

\title{
    Sexuality in the Church \\ 
    \large Historical and Contemporary Investigation}
% /title

\author{
  Asher Mancinelli \\
  Department of Computer Science\\
  Eastern Washington University\\
  Spokane, WA 99218 \\
  \texttt{asher.mancinelli@pnnl.gov} \\ }
% /author

\begin{document}
\maketitle

\begin{abstract}
    The Church has had many views on sex throughout history, and has valued
    differing opinions on sex in many ways. Each generation brings with it
    new strands and beliefs and applications; each is however founded on the 
    strong cornerstone of Scripture and faithful exegesis. Asexuality is
    a contemporary concept, but it's essence can be clearly seen across
    history within the church.
\end{abstract}

% Consider using the keywords section to define terms that I do not 
% undersrtand well. This will help me communicate well with the reader
% even if we have different understandings of terms. We will have to come
% to terms, and since this is a one-way form of communication I have to be 
% the one to define terms, lest ambiguity and misunderstanding permeate the
% paper
\keywords{\textbf{Asexuality}: those who experience no sexual attraction
    or desire \and \textbf{the Church}: the royal sense of the church;
    the global church across time since it's inception \and \textbf{the Early
    Church Fathers}: the most active leaders in the church during the 
    \textit{Patristic era} \and \textbf{the Torah}: the first five books
    of the Jewish scriptures (Genesis, Exodus, Leviticus, Numbers and 
    Deuteronomy) \and \textbf{the Law}: the 613 laws of the Torah
    which the Jews held to \and \textbf{exegesis} critical 
    explanation or interpretation of a text, especially of scripture}
%/keywords

\doublespace

\section{Introduction}
    % Consider writing this after the outline has been finished
    I seek here a broad investigation into the 
    history of sexuality in the church, with special attention paid 
    to asexuality.  Note that any Scriptural reference will be taken 
    from the \textit{English Standard Version} of the Bible unless 
    otherwise specified. Note also that I do not make any claim to a complete
    understanding of the historical church - 
    the underrepresentation of women in the history of the church is no
    secret, and I do
    not anticipate that my research will be able to compensate for this.
    I will however do my best to find resources on women in the church
    through history.

    \par The Church has traditionally
        not had sufficient language to discuss asexuality, so there 
        are challenges 
        in investigating a modern term in an historical light, but this work 
        aims to understand asexuality through the history of the church as
        comprehensively as possible.
    %/par

\section{First Century: In the Wake of Judaism}

    In both Judaism and Christianity, context is the primary determinant in 
    the interpreted morality of any sexual action, and the methods of 
    interpreting context vary greatly in either faith. Here we 
    compare and contrast the 
    methods of determining morality in either faith in the first century. The
    reader should mind the fact that either faith has had a plethora 
    of opinions on sexuality over the last two millenia, and this 
    section only examines some sects of Judaism and the early church in 
    the first century. This narrow investigation does not necessarily 
    reflect contemporary opinion.

\subsection{Jewish Theology on Sexuality in the 1$^{st}$ Century}
    \par Christianity was born out of Jewish culture, therefore Jewish theology
        of sexuality is a necessary primer for understanding Christian theology
        of sexuality. Of the four major sects of Judaism in the first century
        (Pharisees, Sadducees, Essenes and Zealots), Jesus's theology most
        resembled the Pharisees\cite{finkelstein_1929}, so it is their 
        opinions that I have given the most attention to.
        Even within the four major sects of Judaism of the time, there are two
        subcamps we must look at: the Hillel and the Shammai\cite{sprinkle_2016}.
    %/par

    \par The Hillel took a more liberal approach to sexuality and divorce,
        while the Shammai were more conservative, holding more strictly to
        the Law. Among other sexual topics of the day, the pleasurability of
        sexuality was hotly debated, whereas some held that sex was only
        for procreation, others firmly held that sex was supposed to be
        enjoyed and pleasurable. 
    %/par

    \par
        Rabbinic midrash looks especially to the story of the great flood
        in Genesis 6 for guidance on this topic\cite{berk2002}:
        \textit{Now the earth was corrupt in God's sight, and the earth
        was filled with violence. And God saw the earth, and behold,
        it was corrupt, for all flesh had corrupted their way on the earth.
        And God said to Noah, “I have determined to make an end of
        all flesh, for the earth is filled with violence through them.
        Behold, I will destroy them with the earth.} (Gen. 6:11-13\cite{esv2016})
    %/par

    \par
        As an extrapolation and inference from these events, the rabbis 
        found reason to believe other forms of sexuality besides incest,
        adultery and bestiality were wrong and overly sensual. They forbade
        believers from, for example, thinking about another partner while
        having sex with the one, intoxicated intercourse, and having sex
        while not properly loving and caring for him or her in other areas
        of life\cite{berk2002}. This list does not prove exhaustive, but aids 
        in laying out the context of sexual theology into which Jesus came
        and amongst which Jesus taught.
    %/par

\subsection{Jesus on Sex}

    \par For the most part, Jesus affirmed the views of the Pharisees of
        his time, which is interesting since he could so often be seen
        and heard condemning their hypocritical ethics. A few significant
        verses ought to be examined here: firstly, in Matthew 19:12,
        Jesus seems to speak of those who have no sexual desire or
        capability:
    %/par

    \par \textit{For there are eunuchs who have been so from birth, and there are eunuchs who have been made eunuchs by men, and there are eunuchs who have made themselves eunuchs for the sake of the kingdom of heaven. Let the one who is able to receive this receive it.}
    %/par

    \par Jesus also often makes references to the Torah and how it plays into
        the life of a believer, specifically Genesis 1 and 2, where Adam and Eve
        are given and received in marriage: \textit{Therefore a man shall leave his father and his mother and hold fast to his wife, and they shall become one flesh. And the man and his wife were both naked and were not ashamed}
    %/par

    \par It is clear here that Jesus believed that sex was a good thing, just
    as Paul and most of the other Jews of the time believed, yet he also 
    seemed to believe that there are some that are born with a 
    predisposition towards asexuality, topics which Paul would expand upon
    in the subsequent year. Other than this, Jesus left most of the sexual 
    discourse to Paul.

\subsection{Pauline Theology on Sex}

    \par Paul represents an interesting theology: he was formerly the most
        devoted of the Jews, deeply passionate and educated on the Law (in
        the royal Jewish sense, not the social or legal sense) before being
        converted to Christianity. 
        % TODO: scriptural reference needed here
        And, as Jesus declares in the Gospel of Matthew, Chapter 5, no part
        of the Law passes away in Christianity. Thus, Paul's expertise in the
        Law is applied to the New Covenant under Christ, which the New
        Testament preaches. \\
    %/par

    \par Readers from a Western culture must pay special attention to
        translation in interpreting Pauline writings however, because for
        hundreds of years, traslations into Western languages have been shaping
        our beliefs. This is the case with other languages as well, but there
        is a unique attachement to certain translations in Western culture,
        especially in Baptist sects in the United States. \\
    %/par

    \par This is seen especially in 1 Cor. 7:1. The King James Version (KJV)
        reads: \textit{Now concerning the things whereof ye wrote unto me: 
        It is good for a man not to touch a woman.} This may perhaps lead
        the reader to believe that Paul would advocate for total abstinence
        and an abandonement of any sexual activity. However, this would be
        a clear misinterpretation of Paul's intent, and a meager attempt at
        exegesis. In other translations, like the English Standard Version,
        the subject of the sentence is quoted for clarity: \textit{Now
        concerning the matters about which you wrote: "It is good for a
        man not to have sexual relations with a woman".}\cite{esv2016}
        It is clear here
        that Paul is directly addressing what the church in Corinth had
        written him previously\cite{kjv_sex}, and that the former's
        interpretation is unhelpful and potentially hurtful. \\
    %/par

    \par If we continue to read Paul's letter, he \textit{advises and compels}
        those in the church to marry, and even specifically admonishes them
        to have sex regularly as a part of the love and mutual submission
        of a healthy relationship:
        \par \textit{Do not deprive one another, except perhaps by agreement
        for a limited time, that you may devote yourselves to prayer; but
        then come together again, so that Satan may not tempt you because
        of your lack of self control.} 
        \par \textit{Now as a concession, not a command, I say this.
        I wish that all were as I myself am. But each has his own
        gift from God, one of one kind, and one of another.}
        (1 Cor. 7:5, \textit{ESV}\cite{esv2016}) \\
        \par I am struck by the fact that so many hold a stigma in Western
        Christian culture against asexual folk, those not sexually
        interested in anyone, when the most substantive writer of the 
        New Testament wished that all had his \textit{gift} of seemingly
        being asexual: He did not command that all should refrain from sex,
        rather that we should be especially careful to have regular sex with
        our marital partners. How I wish that the church would more
        carefully examine the Scripture before passing a value judgement
        on one who does not experience sexually attraction; they
        consequently miss the fact that Paul willed that they themselves
        would be gifted with asexuality, that they may more completely
        devote themselves to the Kingdom - and asexuality is a gift. \\
    %/par

    \singlespace
    \par To conclude the section on Paul's theology, we draw two conclusions:
        \begin{itemize}
            \item{translation affects interpretation, thus the church ought
                to strive for the most accurate translation possible, and}
            \item{Paul was of the opinion that sex is a gift, and
                asexuality in its own right.}
        \end{itemize}
        \doublespace
        I am sure that had Paul not suffered the heuristical injustice of not
        having language or concept of asexuality, he would have ensured that
        asexuality be recognized as a gift, of greater or equal value to
        that of any other form of sexuality.
    %/par

\section{Early Church Fathers}
    
    \par The void of resources on women in this time period must be recognized
        up front. The fact that the time period is recognized as the \textit{
        Patristic era} should serve as some indicator that female representation
        in this research will be lacking.
    %/par

    \par With that recognized, we now look to the Early Church Fathers'
        theology of sex. It would seem that, as in every era, the Church
        Fathers made some concessions to their contemporaries in their
        theology\cite{ezra}. Much owed to the Gnostics, Manichaeists and 
        Neo-Platonists, the Early Church Fathers downplayed the giftedness
        of sex, favoring the stoic, emotionless resistance of sexual desire.
        As seen in Paul's theology, asexuality and sex itself are both
        gifts, the latter apparently minimized in this era, whereas the
        former is minimized in the present day.
    %/par

\subsection{Greco-Roman Culture}

    Just as the Jewish groundwork had to be laid, we now look to understand
    Greco-Roman culture. In this culture, women had unprecendented
    autonomy, though they were also understood primarily by their relationship
    with the men in their lives, e.g. as a mother, sister, daughter, etc.
    Marital relationships were also not viewed as especially permanent; divorce
    and infidelity were rather frequent\cite{kelly19}. The (male) master of
    the house was socially permitted to do just about whatever he pleased
    within the walls of his house, with whoever he pleased, of those
    under his authority.

\subsection{Pragmatic Theology}

    Much of the writings of the Early Church Fathers was about the practical
    morality of sex: when should it happen? Should it be enjoyed? What is
    it's purpose?

    \par Iranaeus and Augustine represent two voices out of a theological 
        position held by many, that sex is a good gift from God, but we are
        not to be slaves to this gift; we must remain in control of
        ourselves and manage our temples well\cite{kelly19}. The early
        church was also more predisposed to ascetic lifestyles which 
        involved just about every form of abstinence. For example, John
        Chrysostom went as far as to claim that virginity is the 
        lifestyle that God originally intended, and that the church is not
        to fear virginity, but to embrace it as a simpler life without
        the clouding stressors of spouse or children.
    %/par

    \par As the Church spread throughout the Roman Empire, the primary
        gathering place for Christians was the house church, which allowed
        for a fuller recognition of spiritual gifting. This involved
        recognizing women as leaders of gatherings, facilitators of community
        and rightfully having a place at the table and a voice in discussion.
        They were not expected to leave the room for talk or worship, and
        they led roles of mentorship, bringing up new believers into the
        faith. the apostle Paul specifically mentions a woman named Phoebe 
        as a deacon in Romans 16:1-2\cite{esv2016}. She also had the honor
        of being the one to deliver the letters to various churches in the 
        region. This ran quite countercultural, which was accepted in differing
        capacities, and contributed to the sexual ethic of the early church. 
    %/par

    \par This autonomy of women in the early church, as well as the 
        reverence given to the letters of Paul resulted in a positive view
        of asexuality. Thus many in the early church did not feel the societal
        pressures to marry and be understood under the light of their marriage
        or sexual partner or children - they could rather be understood as
        members of their local church, serving each others and their 
        communities. 
    %/par

    \par Those not given to marriage could better devote themselves to their
        local church, and were therefor counterculturally 
        celebrated for their sexual state.
    %/par

\section{Reformation: Sex is Back in Fashion}

    Martin Luther's, as the major driving force of the reformation 
    of the church, contributed much to the formation of the church's new
    sexual ethic. He was even known for sneaking nuns out of their 
    monestaries in large jars so that they would be free to marry and
    join a new post-reformation local church.

    \par Obvious from the nun-sneaking stories, Martin Luther was
        largely in favor of sex and marriage. In the year 1519, Luther
        began writing and delivering sermons praising marriage and sex,
        causing no small measure of hard feelings from his catholic
        contemporaries\cite{luthersex}. He agreed with St. Augustine on
        the purposes for marriage, and wrote about the intense and 
        precious role of children in a marriage. He even took a wife for
        himself from the order of nuns he aided in escaping their
        covenants. His marriage began with little to no sexual desire
        between them, arising primarily from mutual esteem and respect, but
        which grew into deep affection. I must wonder if they would have
        both been asexual singles had they not been such strong advocates
        of marriage in response to the catholic church's resistance to
        marriage. Perhaps if they were together today, they would be
        advocates of healthy marriage, yet would themselves identify
        as close asexual friends. 
    %/par


\section{The 21$^{st}$ Century Christian}

    \par Much of contemporary theology is influenced by the reformation
        and the sexual ethics of Martin Luther and John Calvin, thus 
        the previous section has much to do with this one. This was 
        recognized at the \textit{Centro Culturale Protestante di Milano} 
        in 2010,
        where Laura Ronchi and Debora Spini said this of the relationship
        between the reformation and female empowerment:
        \par \textit{The religious and social change brought about by the 
        Protestant Reformation contributed greatly to changing the position 
        of women in the churches. It also gave them a new awareness 
        that produced other changes in the family and society: it was 
        a long process, aptly described as “crossing the wilderness”. The 
        new message of freedom that spread all over Europe bore with it a 
        new feminine social protagonism. Women had the courage to distrust 
        the religious and political authorities, and to affirm their 
        individuality. Among them was no lack of preachers and writers.}
        \cite{free2013}\cite{lutherfem}
    %/par

    \par This reformation theology started a concern for the theological
        education (and education overall) for women. Martin Luther wrote in
        \textit{The Freedom of a Christian} that 
        \par \textit{The first care of every Christian ought to be to lay aside all reliance on works, and strengthen his faith more and more, and by it grow in the knowledge, not of works, but of Christ Jesus}\cite{lutherfem}\cite{luther31}.
        
    \par This encouraged women then and encourages women now to pursue education
        and be defined not by their sexual status, or by the men in their lives,
        but rather by their care for others and knowledge of the faith. I 
        again am left to wonder how many learned women of that day would
        count themselves among the ranks of asexual 
        Christians of today, that seek
        not the affirmation of the world but of their God, being defined in
        no part by their sex lives. Men had long had the freedom to live
        lives very close to that of the modern educated asexual, yet
        this represented that same freedom for women as well. Women and men 
        alike could now be identified apart from sex, should that be their
        disposition, and were now valued for it.
        This evidence leads me to believe that this detachment from 
        sexuality in identity has led to the modern Christian asexual, and
        more throughout history would have identified as such if they
        had the language and concept for it.
    %/par
    
\section{Conclusion}

    \par From the formation of Christianity out of the legalistic and highly
        contested Jewish sexual ethic to the Reformation's radical break 
        from Catholicism, there have been a plethora of views an sexuality,
        specifically asexuality. Asexuality has been valued differently in
        each age, and each generation's Christianity has unfortunately
        conformed in some capacity to culture, but there has been some
        value attributed to asexuality at every point in the history of 
        the Church.
    %/par

\bibliographystyle{unsrt}  
\bibliography{references}

\end{document}
